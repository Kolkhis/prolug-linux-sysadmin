\documentclass[14pt,compress,usenames,dvipsnames,aspectratio=169]{beamer}
%\usetheme{Singapore}
\useoutertheme{shadow}
\usetheme{CambridgeUS}
\definecolor{mygreen}{RGB}{150, 255, 210}%186}
\definecolor{leftblue}{RGB}{230,255,255}
\definecolor{rightblue}{RGB}{111,195,223}
\definecolor{lefttron}{RGB}{19,44,65}
\definecolor{myblack}{RGB}{27,27,27}
\definecolor{mypurple}{RGB}{205,87,255}

\usecolortheme{owl}

% \setbeamercolor{section in head/foot}{fg = white,bg=black}
\setbeamercolor{title}{fg=mygreen,bg=black}
\setbeamercolor{titlelike}{fg=yellow,bg=black}
\setbeamercolor{item}{fg=mygreen}
\setbeamercolor{block title}{fg=white,bg=myblack!200}
\setbeamercolor{block body}{bg=normal text.bg!80}
\setbeamertemplate{blocks}[rounded][shadow=true]
\setbeamertemplate{headline}{}
\setbeamertemplate{footline}[frame number]
\setbeamercolor{normal text}{fg=white,bg=myblack}%!89.9}

%Gradient
\setbeamercolor{frametitle}{fg=orange,bg=black}
\setbeamercolor{frametitle right}{fg=white,bg=gray}

\usepackage[utf8]{inputenc}
\usepackage{amsmath}
\usepackage{amsfonts}
\usepackage{amssymb}
\usepackage{graphicx}
\usepackage{shadowtext}
\usepackage{multicol}
\usepackage[makeroom]{cancel}

\usepackage{listings} % For code blocks

%\graphicspath{{./figures/},
%}

%\AtBeginSection{\frame{\sectionpage}}

\usepackage{natbib}
\usepackage{float}
\usepackage{subcaption}
\usepackage{xcolor}
\usepackage{natbib}
\usepackage{bibentry}
\usepackage{animate}
\usepackage{varwidth}
\usepackage{appendixnumberbeamer}

\usepackage{tikz}
\usetikzlibrary{shapes,arrows}

%%%%%%%%% TITLE SLIDE %%%%%%%%%
\title{\textbf{Automated Setup of a Monitoring Stack using Ansible}}
\author{Using Proxmox on a Dell PowerEdge R730}

\date{}  % Get rid of date

\usefonttheme{professionalfonts}

\usepackage{mydefs}

\setbeamercovered{transparent} 
\setbeamertemplate{navigation symbols}{} 
\titlegraphic{
\begin{center}
\vspace*{-30pt}

\vspace*{10pt}
    \text{by Kolkhis}
\end{center}
}
%%%%%%%%% END TITLE SLIDE %%%%%%%%%



\begin{document}

\setbeamercovered{invisible}

\begin{frame}[plain]
\titlepage
\end{frame}

%%%% CONTENT FRAMES %%%%

%%%%%%%%%%%%%% FRAME 1: Project Goals %%%%%%%%%%%%%%%%%%%%
\begin{frame}{Goals}
    \begin{enumerate}
        \item{Set up a Type 1 hypervisor (Proxmox) for running Virtual Machines and
            Containerized Applications.} 
        \item{Set up monitoring tools for analyzing resource usage and logs. }
        \item{Automate the setup of those monitoring tools on all VMs using Ansible.}
    \end{enumerate}
\end{frame}
%%%%%%%%%%%%%%%%%%%%%%%%%%%%%%%%%%%%%%%%%%%%%%%


%%%%%%%%%%%%%%%%%%%%%%%%%%%% MONITORING %%%%%%%%%%%%%%%%%%%%%%%%%%%%%%%
%%%%%%%%%%%%%% FRAME Central VM Setup %%%%%%%%%%%%%%%%%%%%
\begin{frame}{Monitoring, Pt1: Create a Centralized VM}
    \begin{enumerate}
        \item{Creating a centralized VM to host the monitoring tools.} 
        \item{All other nodes will send metrics and logs to this VM.}
        \item{I gave this VM 2 CPUs, 6GB RAM, and 80GB disk space.}
    \end{enumerate}
    \begin{figure}
        \centering
        \includegraphics[width=0.75\linewidth]{monitoring_stack_viz.png}
        \caption{Monitoring Schema}
        \label{fig:enter-label}
    \end{figure} 
\end{frame}
%%%%%%%%%%%%%%%%%%%%%%%%%%%%%%%%%%%%%%%%%%%%%%%


%%%%%%%%%%%%%% FRAME Write Playbooks to Install Tools on Central VM %%%%%%%%%%%%%%%%%%%%
\begin{frame}{Monitoring, Pt2: Install the Tools on Central VM}
    Writing an Ansible playbook to install tools on the central VM.
    \begin{enumerate}
        \item{I started with the central VM to host these tools:} 
            \begin{enumerate}
                \item{Prometheus (Metrics TSDB)}
                \item{Grafana (Visualizations)}
                \item{Node_Exporter (Collecting Metrics)}
            \end{enumerate}
        \item{I wrote a series of playbooks that would run on that VM and install these tools.}
    \end{enumerate}
\end{frame}
%%%%%%%%%%%%%%%%%%%%%%%%%%%%%%%%%%%%%%%%%%%%%%%

%%%%%%%%%%%%%% FRAME Prometheus and node_exporter %%%%%%%%%%%%%%%%%%%%
\begin{frame}{Prometheus and Node\_Exporter for System Metrics}
    Promtheus and Node\_Exporter are used for collecting and storing system metrics.
    \begin{enumerate}
        \item{Node\_Exporter gathers system metrics and exposes them on an endpoint
            for Prometheus to scrape.}
        \item{Promtheus scrapes the metrics from targets in its configuration files and stores them.}
        \item{This data is then available for Grafana to display.}
    \end{enumerate}
\end{frame}
%%%%%%%%%%%%%%%%%%%%%%%%%%%%%%%%%%%%%%%%%%%%%%%



%%%%%%%%%%%%%% FRAME Loki and Promtail %%%%%%%%%%%%%%%%%%%%
\begin{frame}{Loki and Promtail for Logging}
    \begin{enumerate}
        \item{Loki and Promtail go together just like Prometheus and Node\_Exporter.  \\}
        \item{While Prometheus and Node\_Exporter are for gathering and storing system metrics,
            Loki and Promtail gather and store system logs.\\}
        \item{Loki is the TSDB, and Promtail collects the logs and sends them to Loki.\\}
        \item{This data is then available for Grafana to display.}
    \end{enumerate}
\end{frame}
%%%%%%%%%%%%%%%%%%%%%%%%%%%%%%%%%%%%%%%%%%%%%%%


%%%%%%%%%%%%%% FRAME Loki and Promtail (cont) %%%%%%%%%%%%%%%%%%%%
\begin{frame}{Loki and Promtail for Logging (cont.)}
    The way the data is transferred between Loki/Promtail is different from Prometheus/Node\_Exporter:
    \begin{enumerate}
        \item{Prometheus/Node\_Exporter:} 
        \begin{enumerate}
            \item{Node\_Exporter makes data available for scraping on the machine it's running on.} 
            \item{Prometheus (the TSDB) goes and scrapes the logs that Node\_Exporter exposes.} 
        \end{enumerate}
        \item{Loki/Promtail:} 
        \begin{enumerate}
            \item{Promtail actively gathers logs \textit{and} sends them to Loki.} 
            \item{Loki accepts the logs from Promtail and stores them.} 
        \end{enumerate}
    \end{enumerate}
\end{frame}
%%%%%%%%%%%%%%%%%%%%%%%%%%%%%%%%%%%%%%%%%%%%%%%



%%%%%%%%%%%%%% FRAME Grafana Playbook %%%%%%%%%%%%%%%%%%%%
\begin{frame}{Grafana Setup Playbook}
The Grafana plabook was simple. It can be installed with most package managers after
    adding the repository.  

    Tasks:
    \begin{enumerate}
        \item{Pull the Grafana GPG key.} 
        \item{Add the Grafana repository, signed by the GPG key.}
        \item{Install using the host OS's package manager.}
        \item{Add a task that starts/enables the Grafana-server systemd service.}
        \item{Use blocks with conditions to detect the OS family to determine where
            to store the GPG key and which package manager to use.}
            %whether to use either dnf/yum or apt.}
        \item{Expose port 3000 if firewalld is running.}
    \end{enumerate}
\end{frame}
%%%%%%%%%%%%%%%%%%%%%%%%%%%%%%%%%%%%%%%%%%%%%%%


%%%%%%%%%%%%%% FRAME Provisioning Datasources && dashboards %%%%%%%%%%%%%%%%%%%%
\begin{frame}{Provisioning Data Sources and Dashboards}
    \begin{enumerate}
        \item{Default dashboards and datasources can be stored in json format.}
        \item{Configure the data sources directly in yaml format in
            \texttt{/etc/grafana/provisioning/datasources/} }
        \item{Configure the dashboards in json format, using a yaml config to 
                specify the path to the dashboard files in 
                \texttt{/etc/grafana/provisioning/datasources/} }

    \end{enumerate}
\end{frame}
%%%%%%%%%%%%%%%%%%%%%%%%%%%%%%%%%%%%%%%%%%%%%%%

%%%%%%%%%%%%%% FRAME Provisioning Configuration Screenshots %%%%%%%%%%%%%%%%%%%%
% Data Source Provisioning File
\begin{figure}
    \centering
    \includegraphics[width=0.55\linewidth]{grafana_datasource_provisioning_file.png}
    \caption{Grafana Datasource Provisioning File}
    \label{fig:enter-label}
\end{figure}

% Dashboard Provisioning File
\begin{figure}
    \centering
    \includegraphics[width=0.55\linewidth]{grafana_dashboard_provisioning_example.png}
    \caption{Grafana Dashboard Provisioning File}
    \label{fig:enter-label}
\end{figure}

%%%%%%%%%%%%%% FRAME Prometheus Playbook %%%%%%%%%%%%%%%%%%%%
\begin{frame}{Prometheus Setup Playbook}
    Tasks:
    \begin{enumerate}
        \item{Pull the tarball from Github.} 
        \item{Create necessary directories and user account.} 
        \item{Extract tarball into \texttt{/var/lib/prometheus} } 
        \item{Copy the binary and service files to their respective directories.} 
        \item{Create (or copy) some default configuration files.} 
        \item{Expose port 9090 if accessing externally.} 
    \end{enumerate}
\end{frame}
%%%%%%%%%%%%%%%%%%%%%%%%%%%%%%%%%%%%%%%%%%%%%%%

%%%%%%%%%%%%%% FRAME Prometheus File-based Service Discovery %%%%%%%%%%%%%%%%%%%% 
\begin{frame}{Prometheus Service Discovery}
    Prometheus supports file-based service discovery.  
    To set it up:
    \begin{enumerate}
        \item{Create a default \texttt{targets.json}} 
        \item{Add a \texttt{file\_sd\_configs} entry in \texttt{/etc/prometheus/prometheus.yml}
            (Prometheus configuration file)} 
        \item{Point the \texttt{file\_sd\_configs} to your targets.json file.} 
        \item{Whenever a change is made in targets.json, Prometheus will detect the
            change and update accordingly.}
    \end{enumerate}
\end{frame}
%%%%%%%%%%%%%%%%%%%%%%%%%%%%%%%%%%%%%%%%%%%%%%%

% TODO: Add a screenshot of file_sd_config in prometheus.yml


%%%%%%%%%%%%%% FRAME Node Exporter Playbook %%%%%%%%%%%%%%%%%%%% 
\begin{frame}{Node Exporter Playbook}
    This process was similar to that of Prometheus.  

    Tasks:
    \begin{enumerate}
        \item{Create user account for \texttt{node\_exporter}.} 
        \item{Download and extract the node\_exporter tarball.} 
        \item{Clone the default config\_files repository for the service files.} 
        \item{Copy binary to \texttt{/usr/sbin}.} 
        \item{Copy service files to \texttt{/etc/systemd/system/}. } 
        \item{Read targets.json from the Prometheus node and append the new target.} 
        \item{If firewalld is running, expose port 9100. } 
        \item{Use systemctl to start/enable node\_exporter.} 
    \end{enumerate}
\end{frame}
%%%%%%%%%%%%%%%%%%%%%%%%%%%%%%%%%%%%%%%%%%%%%%%


%%%%%%%%%%%%%% FRAME Node Exporter Targets %%%%%%%%%%%%%%%%%%%%
\begin{frame}{Node Exporter Playbook, Pt2}
    Benefits of using file-based service discovery:
    \begin{enumerate}
        \item{Every time this playbook deploys node\_exporter on a host, it adds 
            that host to targets.json (/etc/prometheus/target.json).} 
        \item{Prometheus will detect the change in this file and update.} 
        \item{This allows for dynamic updates of Prometheus scrape targets without
            having to restart the Prometheus service.} 
    \end{enumerate}
\end{frame}
%%%%%%%%%%%%%%%%%%%%%%%%%%%%%%%%%%%%%%%%%%%%%%%


%%%%%%%%%%%%%% FRAME Appending to Targets.json %%%%%%%%%%%%%%%%%%%%
\begin{frame}{Dynamically Adding Prometheus Targets}
    In order to append to the list of Prometheus targets, I had to read it in as a
    Python data structure.  
    \begin{enumerate}
        \item{Read in the contents of \texttt{targets.json} with
            \texttt{ansible.builtin.slurp}.  } 
        \item{Decode the raw data. Slurp reads files in as a base64-encoded string.
            \texttt{targets\_file\_contents.content | b64decode | from\_json} } 
        \item{This is now a valid python data structure that can be manipulated. } 
        \item{Append it to the list like you normally would in python, formatted the
            way \texttt{json} likes.  }
              
    \end{enumerate}
\end{frame}
%%%%%%%%%%%%%%%%%%%%%%%%%%%%%%%%%%%%%%%%%%%%%%%




\begin{frame}{References}
    \bibliographystyle{apalike}
        Proxmox docs: https://pve.proxmox.com/wiki/  
        Ansible docs: https://docs.ansible.com/ansible/latest/collections/ansible/builtin  
        ProLUG Labs on Killercoda: https://killercoda.com/het-tanis/course/Linux-Labs  
        Prometheus File-based Service Discovery Config https://prometheus.io/docs/prometheus/latest/configuration/configuration/#file_sd_config  
    \bibliography{bib}
\end{frame}

\end{document}
